\documentclass[
    ngerman,
    color=1b,
    load_common,
    % submission,
    leqno,
    dark_mode,
    boxarc,
    % shell_escape=false,
    solution=true,
]{rubos-tuda-template}


% --Definitionen für das Dokument (alle Optional)--
% \sheetnumber{1}
\version{1.0 Snapshot}
\semester{Sommersemester 2021}
\topic{Analysis und Stochastik}
%  \fachbereich{Mathematik}
\author{Ruben Deisenroth\and Simon Hüner}
% \contributor{Simon Hüner}
\date{\today}

% --Stilistische Anpassungen für die Klausur--
\ExplSyntaxOn
% \renewcommand*{\printAuthor}[1]{
%   #1
% }
\tl_gset:Nn \g_ptxcd_logofile_tl {}

\ExplSyntaxOff
\renewcommand*{\taskformat}{\taskname\tasksep\thetask{}} % Taskprefix
\renewcommand*{\subtaskformat}{\thesubtask\enskip}
\renewcommand{\printTopic}{\fatsf{Themen:}~\getTopic{}}
% Titelbereich
\termStyle{left-right-manual}
\termLeft{printAuthor,printContributor,printTopic}
\termRight{printSemester,printVersion}
\term{\fatsf{Bearbeitungszeit:} 90 Minuten\\\printDate{}\hfill\printFachbereich{}}
% Kopfzeile
\ConfigureHeadline{
	headline={aud}
}

% Dokumentenmakros
\RequirePackage{pifont}
\newcommand{\tickedbox}{\huge\rlap{$\square$}{\Huge\hspace{1pt}\ding{55}}}
\newcommand{\redtickedbox}{\huge\rlap{$\square$}{\Huge\hspace{1pt}\color{red}\ding{55}}}
\newcommand{\untickedbox}{\huge\rlap{$\square$}}

% --Beginn Dokument--

\begin{document}
\title[Mathe Klausur]{Übungsklausur\\im Fach Mathematik}
\maketitle{}
\newcommand{\titledistance}{\\[1em]} % Distance for title text frame
\newcommand{\unchecked}{\rlap{\(\square\)}} % Checkbox
\begin{minipage}[t]{0.45 \textwidth}
    Name: \dotfill
    \titledistance
    Vorname: \dotfill
    \titledistance
    Matrikelnummer: \dotfill
    \titledistance
    Unterschrift: \dotfill
\end{minipage}
\hfill\vline\hfill
\begin{minipage}[t]{0.45 \textwidth}
    Studiengang: \dotfill
    \titledistance
    Semester: \dotfill
    \titledistance
    Anzahl Abgegebener Zusatzblätter: \dotfill
    \titledistance
    Drittversuch? Bitte ggf. ankreuzen \hspace{0.5cm} \huge\unchecked\normalsize
\end{minipage}
% Punktetabelle

\begin{figure}[ht]
    \centering
    \begin{tikzpicture}
        \tikzstyle{gradetablenode}=[minimum size=1cm, draw]
        \tikzstyle{lightgraygradetablenode}=[gradetablenode, fill=fgcolor!10!\thepagecolor]
        \node[lightgraygradetablenode, text width=3cm, font=\sffamily, anchor=west] (n0-0) at (0, 1){Aufgabe};
        \node[gradetablenode, text width=3cm, font=\sffamily, below right = -\pgflinewidth and 0cm of n0-0.south west] (n1-0){Punkte (max)};
        \node[gradetablenode, text width=3cm, font=\sffamily, below right = -\pgflinewidth and 0cm of n1-0.south west] (n2-0){Punkte (erreicht)};
        \xdef\lasttask{0}
        \mapTasks{
            \node[lightgraygradetablenode, font=\bfseries, right=-\pgflinewidth of n0-\the\numexpr\value{task}-1\relax.east] (n0-\the\value{task}) {\thetask{}};
            \node[gradetablenode, right=-\pgflinewidth of n1-\the\numexpr\value{task}-1\relax.east] (n1-\the\value{task}) {\getPoints{\thetask}};
            \node[gradetablenode, right=-\pgflinewidth of n2-\the\numexpr\value{task}-1\relax.east] (n2-\the\value{task}) {};
            \xdef\lasttask{\numexpr\lasttask+1\relax}
        }
        % Probably + 1 in node names is too much, will check later
        \node[lightgraygradetablenode, minimum width=1.1cm, right=-\pgflinewidth of n0-\the\numexpr\lasttask\relax.east] (n0-\the\numexpr\lasttask + 1\relax){$\boldsymbol{\Sigma}$};
        \node[gradetablenode, minimum width=1.1cm, right=-\pgflinewidth of n1-\the\numexpr\lasttask\relax.east] (n1-\the\numexpr\lasttask + 1\relax){$\getPointsTotal{}$};
        \node[gradetablenode, minimum width=1.1cm, right=-\pgflinewidth of n2-\the\numexpr\lasttask\relax.east] (n2-\the\numexpr\lasttask + 1\relax){};
        % % Extra lines
        \draw[transform canvas={xshift = -1mm}] (n0-0.north east) -- (n2-0.south east);
        \draw[transform canvas={xshift = 1mm}] (n0-\the\numexpr\lasttask+1.north west) -- (n2-\the\numexpr\lasttask+1.south west);
        \draw[thick] (n0-0.north west) rectangle (n2-\the\numexpr\lasttask+1.south east);
    \end{tikzpicture}
    % \caption*{Punkteübersicht}
\end{figure}
{\large \textbf{Hinweise:}}
\begin{itemize}
    \item Überprüfen Sie zunächst, ob Ihre Klausur die Seiten 1 bis \pageref*{LastPage} besitzt.
    \item Füllen Sie zuerst das Deckblatt aus und halten Sie Studienausweis und Lichtbildausweis bereit.
    \item Sie sollten die Lösungen möglichst direkt in die Klausur eintragen. Reicht der vorhandene Platz nicht aus, so können Sie zusätzliche Blätter verwenden, die Sie zuerst mit Nachname, Vorname und Matrikelnummer kennzeichnen.
    \item Nicht mit Bleistift schreiben und keine roten oder grünen Stifte verwenden.
    \item Erlaubte Hilfsmittel: \textbf{Taschenrechner}, \textbf{Formelsamlung}
\end{itemize}
{\large \textbf{Viel Erfolg!}}
\clearpage



%% --Beginn Hausübung--%%
\section*{Analysis (Differenzialrechnung)}
% Aufgabe 1
\begin{task}[points=1]{Differenzen- und Diffenrenzialquotient anwenden}

\end{task}
% Aufgabe 2
\begin{task}[points=1]{Ganzrationale Funktionen ableiten (Potenz-, Faktor- und Summenregel)}

\end{task}
\clearpage
% Aufgabe 3
\begin{task}[points=6]{Zusammenhang zwischen Funktion und Ableitungsfunktion begründen}
    % Koordinatensystemgröße für Gesamte Aufgabe
    \def\coordinatesystemradius{4}

    % Originalgröße
    % \begin{tikzpicture}
    %     \begin{axis}[
    %         xlabel=$x$,
    %         ylabel={$y$},
    %         axis lines=middle,
    %         axis line style={-{Triangle}{Bar}, thick},
    %         xmin=-\the\dimexpr\coordinatesystemradius pt + .5pt\relax,
    %         xmax=\the\dimexpr\coordinatesystemradius pt + .5pt\relax,
    %         ymin=-\the\dimexpr\coordinatesystemradius pt + .5pt\relax,
    %         ymax=\the\dimexpr\coordinatesystemradius pt + .5pt\relax,
    %         x=1cm,
    %         y=1cm,
    %         grid=both,
    %         minor tick num=1,
    %         xtick={-\coordinatesystemradius,...,\coordinatesystemradius},
    %         ytick={-\coordinatesystemradius,...,\coordinatesystemradius},
    %         every tick/.style={color=fgcolor},
    %         major grid style = {line width=.8pt},
    %         ]
    %         \addplot[smooth, draw=accentcolor,very thick]{x^2-2};
    %     \end{axis}
    % \end{tikzpicture}

    %50%
    % \begin{tikzpicture}
    %     \def\coordinatesystemsize{\coordinatesystemradius}
    %     % Odd System Size
    %     \ifthenelse{\isodd{\coordinatesystemsize}}{}{
    %         \def\coordinatesystemsize{\the\numexpr\coordinatesystemradius + 1\relax}
    %     }
    %     \begin{axis}[
    %         xlabel=$x$,
    %         ylabel={$y$},
    %         axis lines=middle,
    %         axis line style={-{Triangle}{Bar}, thick},
    %         xmin=-\coordinatesystemsize,
    %         xmax=\coordinatesystemsize,
    %         ymin=-\coordinatesystemsize,
    %         ymax=\coordinatesystemsize,
    %         x=.5cm,
    %         y=.5cm,
    %         grid=both,
    %         minor tick num=1,
    %         xtick distance={2},
    %         ytick distance={2},
    %         every tick/.style={color=fgcolor},
    %         major grid style = {line width=.8pt},
    %         ]
    %         \addplot[smooth, draw=accentcolor,very thick]{x^2-2};
    %     \end{axis}
    % \end{tikzpicture}

    \begin{grayInfoBox}
        Kreuzen sie bei den Graphen $f(x)$, $g(x)$ und $h(x)$ jeweils die richtige Ableitung an. \textbf{Begründen} Sie Ihre Antwort
    \end{grayInfoBox}
    \begin{figure}[ht]
        \centering
        %f(x)
        \begin{subfigure}[t]{.24\textwidth}
            \centering
            \begin{tikzpicture}
                \def\coordinatesystemsize{\coordinatesystemradius}
                % Odd System Size
                % \ifthenelse{\isodd{\coordinatesystemsize}}{}{
                %     \def\coordinatesystemsize{\the\numexpr\coordinatesystemradius + 1\relax}
                % }
                \begin{axis}[
                    xlabel=$x$,
                    ylabel={$y$},
                    axis lines=middle,
                    axis line style={-{Triangle}{Bar}, thick},
                    xmin=-\coordinatesystemsize,
                    xmax=\coordinatesystemsize,
                    ymin=-\coordinatesystemsize,
                    ymax=\coordinatesystemsize,
                    x=.5cm,
                    y=.5cm,
                    grid=both,
                    minor tick num=1,
                    xtick distance={2},
                    ytick distance={2},
                    every tick/.style={color=fgcolor},
                    major grid style = {line width=.8pt},
                    ]
                    \addplot[smooth, samples=100, draw=accentcolor,very thick]{x^2-2};
                \end{axis}
            \end{tikzpicture}
            \caption*{$f(x)$}
        \end{subfigure}
        \begin{subfigure}[t]{.25\textwidth}
            \centering
            \begin{tikzpicture}
                \def\coordinatesystemsize{\coordinatesystemradius}
                % Odd System Size
                % \ifthenelse{\isodd{\coordinatesystemsize}}{}{
                %     \def\coordinatesystemsize{\the\numexpr\coordinatesystemradius + 1\relax}
                % }
                \begin{axis}[
                    xlabel=$x$,
                    ylabel={$y$},
                    axis lines=middle,
                    axis line style={-{Triangle}{Bar}, thick},
                    xmin=-\coordinatesystemsize,
                    xmax=\coordinatesystemsize,
                    ymin=-\coordinatesystemsize,
                    ymax=\coordinatesystemsize,
                    x=.5cm,
                    y=.5cm,
                    grid=both,
                    minor tick num=1,
                    xtick distance={2},
                    ytick distance={2},
                    every tick/.style={color=fgcolor},
                    major grid style = {line width=.8pt},
                    ]
                    \addplot[smooth, samples=100, draw=red,very thick]{x^2};
                \end{axis}
            \end{tikzpicture}
            \caption*{\untickedbox}
        \end{subfigure}
        \begin{subfigure}[t]{.25\textwidth}
            \centering
            \begin{tikzpicture}
                \def\coordinatesystemsize{\coordinatesystemradius}
                % Odd System Size
                % \ifthenelse{\isodd{\coordinatesystemsize}}{}{
                %     \def\coordinatesystemsize{\the\numexpr\coordinatesystemradius + 1\relax}
                % }
                \begin{axis}[
                    xlabel=$x$,
                    ylabel={$y$},
                    axis lines=middle,
                    axis line style={-{Triangle}{Bar}, thick},
                    xmin=-\coordinatesystemsize,
                    xmax=\coordinatesystemsize,
                    ymin=-\coordinatesystemsize,
                    ymax=\coordinatesystemsize,
                    x=.5cm,
                    y=.5cm,
                    grid=both,
                    minor tick num=1,
                    xtick distance={2},
                    ytick distance={2},
                    every tick/.style={color=fgcolor},
                    major grid style = {line width=.8pt},
                    ]
                    \addplot[smooth, samples=100, draw=red,very thick]{2*x};
                \end{axis}
            \end{tikzpicture}
            \caption*{\IfSolutionTF{\redtickedbox}{\untickedbox}}
        \end{subfigure}
        \begin{subfigure}[t]{.24\textwidth}
            \centering
            \begin{tikzpicture}
                \def\coordinatesystemsize{\coordinatesystemradius}
                % Odd System Size
                % \ifthenelse{\isodd{\coordinatesystemsize}}{}{
                %     \def\coordinatesystemsize{\the\numexpr\coordinatesystemradius + 1\relax}
                % }
                \begin{axis}[
                    xlabel=$x$,
                    ylabel={$y$},
                    axis lines=middle,
                    axis line style={-{Triangle}{Bar}, thick},
                    xmin=-\coordinatesystemsize,
                    xmax=\coordinatesystemsize,
                    ymin=-\coordinatesystemsize,
                    ymax=\coordinatesystemsize,
                    x=.5cm,
                    y=.5cm,
                    grid=both,
                    minor tick num=1,
                    xtick distance={2},
                    ytick distance={2},
                    every tick/.style={color=fgcolor},
                    major grid style = {line width=.8pt},
                    ]
                    \addplot[smooth, samples=100, draw=red,very thick]{-2*x};
                \end{axis}
            \end{tikzpicture}
            \caption*{\untickedbox}
        \end{subfigure}
        %g(x)
        \begin{subfigure}[t]{.24\textwidth}
            \centering
            \begin{tikzpicture}
                \def\coordinatesystemsize{\coordinatesystemradius}
                % Odd System Size
                % \ifthenelse{\isodd{\coordinatesystemsize}}{}{
                %     \def\coordinatesystemsize{\the\numexpr\coordinatesystemradius + 1\relax}
                % }
                \begin{axis}[
                    xlabel=$x$,
                    ylabel={$y$},
                    axis lines=middle,
                    axis line style={-{Triangle}{Bar}, thick},
                    xmin=-\coordinatesystemsize,
                    xmax=\coordinatesystemsize,
                    ymin=-\coordinatesystemsize,
                    ymax=\coordinatesystemsize,
                    x=.5cm,
                    y=.5cm,
                    grid=both,
                    minor tick num=1,
                    xtick distance={2},
                    ytick distance={2},
                    every tick/.style={color=fgcolor},
                    major grid style = {line width=.8pt},
                    ]
                    \addplot[smooth, samples=100, draw=accentcolor,very thick]{(x-1)^3+2};
                \end{axis}
            \end{tikzpicture}
            \caption*{$g(x)$}
        \end{subfigure}
        \begin{subfigure}[t]{.25\textwidth}
            \centering
            \begin{tikzpicture}
                \def\coordinatesystemsize{\coordinatesystemradius}
                % Odd System Size
                % \ifthenelse{\isodd{\coordinatesystemsize}}{}{
                %     \def\coordinatesystemsize{\the\numexpr\coordinatesystemradius + 1\relax}
                % }
                \begin{axis}[
                    xlabel=$x$,
                    ylabel={$y$},
                    axis lines=middle,
                    axis line style={-{Triangle}{Bar}, thick},
                    xmin=-\coordinatesystemsize,
                    xmax=\coordinatesystemsize,
                    ymin=-\coordinatesystemsize,
                    ymax=\coordinatesystemsize,
                    x=.5cm,
                    y=.5cm,
                    grid=both,
                    minor tick num=1,
                    xtick distance={2},
                    ytick distance={2},
                    every tick/.style={color=fgcolor},
                    major grid style = {line width=.8pt},
                    ]
                    \addplot[smooth, samples=100, draw=red,very thick]{3*(x-1)^2};
                \end{axis}
            \end{tikzpicture}
            \caption*{\IfSolutionTF{\redtickedbox}{\untickedbox}}
        \end{subfigure}
        \begin{subfigure}[t]{.25\textwidth}
            \centering
            \begin{tikzpicture}
                \def\coordinatesystemsize{\coordinatesystemradius}
                % Odd System Size
                % \ifthenelse{\isodd{\coordinatesystemsize}}{}{
                %     \def\coordinatesystemsize{\the\numexpr\coordinatesystemradius + 1\relax}
                % }
                \begin{axis}[
                    xlabel=$x$,
                    ylabel={$y$},
                    axis lines=middle,
                    axis line style={-{Triangle}{Bar}, thick},
                    xmin=-\coordinatesystemsize,
                    xmax=\coordinatesystemsize,
                    ymin=-\coordinatesystemsize,
                    ymax=\coordinatesystemsize,
                    x=.5cm,
                    y=.5cm,
                    grid=both,
                    minor tick num=1,
                    xtick distance={2},
                    ytick distance={2},
                    every tick/.style={color=fgcolor},
                    major grid style = {line width=.8pt},
                    ]
                    \addplot[smooth, samples=100, draw=red,very thick]{2*(x-1)};
                \end{axis}
            \end{tikzpicture}
            \caption*{\untickedbox}
        \end{subfigure}
        \begin{subfigure}[t]{.24\textwidth}
            \centering
            \begin{tikzpicture}
                \def\coordinatesystemsize{\coordinatesystemradius}
                % Odd System Size
                % \ifthenelse{\isodd{\coordinatesystemsize}}{}{
                %     \def\coordinatesystemsize{\the\numexpr\coordinatesystemradius + 1\relax}
                % }
                \begin{axis}[
                    xlabel=$x$,
                    ylabel={$y$},
                    axis lines=middle,
                    axis line style={-{Triangle}{Bar}, thick},
                    xmin=-\coordinatesystemsize,
                    xmax=\coordinatesystemsize,
                    ymin=-\coordinatesystemsize,
                    ymax=\coordinatesystemsize,
                    x=.5cm,
                    y=.5cm,
                    grid=both,
                    minor tick num=1,
                    xtick distance={2},
                    ytick distance={2},
                    every tick/.style={color=fgcolor},
                    major grid style = {line width=.8pt},
                    ]
                    \addplot[smooth, samples=100, draw=red,very thick]{-3*(x-1)^2+2};
                \end{axis}
            \end{tikzpicture}
            \caption*{\untickedbox}
        \end{subfigure}
        %h(x)
        \begin{subfigure}[t]{.24\textwidth}
            \centering
            \begin{tikzpicture}
                \def\coordinatesystemsize{\coordinatesystemradius}
                % Odd System Size
                % \ifthenelse{\isodd{\coordinatesystemsize}}{}{
                %     \def\coordinatesystemsize{\the\numexpr\coordinatesystemradius + 1\relax}
                % }
                \begin{axis}[
                    xlabel=$x$,
                    ylabel={$y$},
                    axis lines=middle,
                    axis line style={-{Triangle}{Bar}, thick},
                    xmin=-\coordinatesystemsize,
                    xmax=\coordinatesystemsize,
                    ymin=-\coordinatesystemsize,
                    ymax=\coordinatesystemsize,
                    x=.5cm,
                    y=.5cm,
                    % restrict y to domain=-100:100,
                    grid=both,
                    minor tick num=1,
                    xtick distance={2},
                    ytick distance={2},
                    every tick/.style={color=fgcolor},
                    major grid style = {line width=.8pt},
                    ]
                    \addplot[smooth, samples=100, samples=100, draw=accentcolor,very thick]{.2 * (x)^2 * (x-2) * (x+2) - 2};
                \end{axis}
            \end{tikzpicture}
            \caption*{$h(x)$}
        \end{subfigure}
        \begin{subfigure}[t]{.25\textwidth}
            \centering
            \begin{tikzpicture}
                \def\coordinatesystemsize{\coordinatesystemradius}
                % Odd System Size
                % \ifthenelse{\isodd{\coordinatesystemsize}}{}{
                %     \def\coordinatesystemsize{\the\numexpr\coordinatesystemradius + 1\relax}
                % }
                \begin{axis}[
                    xlabel=$x$,
                    ylabel={$y$},
                    axis lines=middle,
                    axis line style={-{Triangle}{Bar}, thick},
                    xmin=-\coordinatesystemsize,
                    xmax=\coordinatesystemsize,
                    ymin=-\coordinatesystemsize,
                    ymax=\coordinatesystemsize,
                    x=.5cm,
                    y=.5cm,
                    grid=both,
                    minor tick num=1,
                    xtick distance={2},
                    ytick distance={2},
                    every tick/.style={color=fgcolor},
                    major grid style = {line width=.8pt},
                    ]
                    \addplot[smooth, samples=100, draw=red,very thick]{x^3};
                \end{axis}
            \end{tikzpicture}
            \caption*{\untickedbox}
        \end{subfigure}
        \begin{subfigure}[t]{.25\textwidth}
            \centering
            \begin{tikzpicture}
                \def\coordinatesystemsize{\coordinatesystemradius}
                % Odd System Size
                % \ifthenelse{\isodd{\coordinatesystemsize}}{}{
                %     \def\coordinatesystemsize{\the\numexpr\coordinatesystemradius + 1\relax}
                % }
                \begin{axis}[
                    xlabel=$x$,
                    ylabel={$y$},
                    axis lines=middle,
                    axis line style={-{Triangle}{Bar}, thick},
                    xmin=-\coordinatesystemsize,
                    xmax=\coordinatesystemsize,
                    ymin=-\coordinatesystemsize,
                    ymax=\coordinatesystemsize,
                    x=.5cm,
                    y=.5cm,
                    grid=both,
                    minor tick num=1,
                    xtick distance={2},
                    ytick distance={2},
                    every tick/.style={color=fgcolor},
                    major grid style = {line width=.8pt},
                    ]
                    \addplot[smooth, samples=100, draw=red,very thick]{-(0.8 * x^3 - 1.6 * x)};
                \end{axis}
            \end{tikzpicture}
            \caption*{\untickedbox}
        \end{subfigure}
        \begin{subfigure}[t]{.24\textwidth}
            \centering
            \begin{tikzpicture}
                \def\coordinatesystemsize{\coordinatesystemradius}
                % Odd System Size
                % \ifthenelse{\isodd{\coordinatesystemsize}}{}{
                %     \def\coordinatesystemsize{\the\numexpr\coordinatesystemradius + 1\relax}
                % }
                \begin{axis}[
                    xlabel=$x$,
                    ylabel={$y$},
                    axis lines=middle,
                    axis line style={-{Triangle}{Bar}, thick},
                    xmin=-\coordinatesystemsize,
                    xmax=\coordinatesystemsize,
                    ymin=-\coordinatesystemsize,
                    ymax=\coordinatesystemsize,
                    x=.5cm,
                    y=.5cm,
                    grid=both,
                    minor tick num=1,
                    xtick distance={2},
                    ytick distance={2},
                    every tick/.style={color=fgcolor},
                    major grid style = {line width=.8pt},
                    ]
                    \addplot[smooth, samples=100, draw=red,very thick]{0.8 * x^3 - 1.6 * x};
                \end{axis}
            \end{tikzpicture}
            \caption*{\IfSolutionTF{\redtickedbox}{\untickedbox}}
        \end{subfigure}
    \end{figure}
    \vspace{-1em}
    $f(x)$:~
    \IfSolutionT{\rlap{\raisebox{2pt}{\color{red}~Ableitung 1 ist stets positiv, aber $f$ fällt für $x<0$. Ableitung 3 ist für $x<0$ positiv, aber $f$ fällt für $x<0$.}}}
    \dotfill\\[1em]\mbox{}\dotfill\\[1em]
    $g(x)$:~
    \IfSolutionT{\rlap{\raisebox{2pt}{\color{red}~Ableitung $2$ und $3$ haben negative $y$ werte, aber $g$ ist streng monoton steigend.}}}
    \dotfill\\[1em]\mbox{}\dotfill\\[1em]
    $h(x)$:~
    \IfSolutionT{\rlap{\raisebox{2pt}{\color{red}~$h$ hat ein lokales Maximum bei $x=0$. Also muss bei $x=0$ ein VZW der Ableitung von \enquote{$+$} nach \enquote{$-$} stattfinden.}}}
    \dotfill\\[1em]\mbox{}\dotfill
\end{task}
\clearpage
% Aufgabe 4
\begin{task}[points=2]{Graph der Ableitungsfunktion einer gegebenen Funktion zeichnen}
    Zeichnen sie in die gegbenen Koordinatensysteme jeweils den Graphen der Ableitungsfunktion $f'(x)$, welche zur dort bereits eingezeichneten Graphen der Funktion $f(x)$ gehöhrt. 
    Achten Sie dabei inbesondere auf die charakteristischen Punkte und eine saubere Stiftführung. Den weißen Bereich am Rechtenseitenrand können sie für Notizen und Überlegungen nutzen.
    \par
    \begin{tikzpicture}
        \begin{axis}[
            xmin=-5.2,
            ymin=-5.2,
            xmax=5.4,
            ymax=5.4,
            width=10.6cm,
            height=10.6cm,
            scale only axis,
            grid=both,
            axis x line=middle,
            axis y line=center,
            minor tick num=1,
            ticklabel style = {font=\footnotesize},
            y axis line style ={-{Triangle[angle=45:0.4em]}},
            xtick={-5,...,5},
            ytick={-5,...,5},
            x axis line style ={-{Triangle[angle=45:0.4em]}},
            axis line style={thick},
            tick style = {thick, fgcolor},
            ylabel={$y$},
            xlabel={$x$}
        ]
        \addplot[thick,samples=100, blue]{0.75*x^2-2.25*x-2};%
        \node[font=\small, blue]at(3.9cm,9.7cm){$f(x)$};
        \end{axis}
        \node[font=\small, anchor=north east] at (0cm,10.6cm){a)};
    \end{tikzpicture}
    \par
    \begin{tikzpicture}
        \begin{axis}[
            xmin=-5.2,
            ymin=-5.2,
            xmax=5.4,
            ymax=5.4,
            width=10.6cm,
            height=10.6cm,
            scale only axis,
            grid=both,
            axis x line=middle,
            axis y line=center,
            minor tick num=1,
            ticklabel style = {font=\footnotesize},
            y axis line style ={-{Triangle[angle=45:0.4em]}},
            xtick={-5,...,5},
            ytick={-5,...,5},
            x axis line style ={-{Triangle[angle=45:0.4em]}},
            axis line style={thick},
            tick style = {thick, fgcolor},
            ylabel={$y$},
            xlabel={$x$}
        ]
        \addplot[thick,samples=100, green!50!black]{(0.05*x^4+(1/15)*x^3-0.6*x^2)-1};%
        \node[font=\small, green!50!black]at(0.8cm,9.7cm){$f(x)$};
        \end{axis}
        \node[font=\small, anchor=north east] at (0cm,10.6cm){b)};
    \end{tikzpicture}
\end{task}
% Aufgabe 5
\begin{task}[points=3]{Funktionsterme und Funktionsgraphen zuordnen}
    
\end{task}
\clearpage
% Aufgabe 6
\begin{task}[points=1]{Aufstellen von Tangenten und Normalen}
    \def\coordinatesystemradius{4}
    \begin{minipage}[t]{.5\textwidth}
        Gegeben sei der Graph $f(x):=x^3+1,5$. \textbf{Berechnen} Sie die Tangente $t_f$ bei $x=1$. Geben Sie auch den Schnittpunkt $p$ der Tangente mit $f$ an und zeichnen Sie die Tangente in das Diagramm ein.\\[2em]
        $t_f=$~
        \IfSolutionT{\rlap{\raisebox{2pt}{\color{red}~$3\cdot (x-1)+1,5$}}}
        \dotfill\\[1em]
        Platz für Rechnungen:
    \end{minipage}
    \begin{minipage}[t]{.49\textwidth}
        \mbox{}\vspace*{-1em}\\
        \begin{tikzpicture}
            \centering
                \begin{axis}[
                    xlabel=$x$,
                    ylabel={$y$},
                    axis lines=middle,
                    axis line style={-{Triangle}{Bar}, thick},
                    xmin=-\the\dimexpr\coordinatesystemradius pt + .5pt\relax,
                    xmax=\the\dimexpr\coordinatesystemradius pt + .5pt\relax,
                    ymin=-\the\dimexpr\coordinatesystemradius pt + .5pt\relax,
                    ymax=\the\dimexpr\coordinatesystemradius pt + .5pt\relax,
                    x=1cm,
                    y=1cm,
                    grid=both,
                    minor tick num=1,
                    xtick={-\coordinatesystemradius,...,\coordinatesystemradius},
                    ytick={-\coordinatesystemradius,...,\coordinatesystemradius},
                    every tick/.style={color=fgcolor},
                    major grid style = {line width=.8pt},
                    ]
                    \addplot[smooth, draw=accentcolor,very thick]{x^3+0.5};
                    \IfSolutionT{
                        \addplot[smooth, draw=red,very thick]{3 * (x-1)+1.5};
                        \node [label={0:{(1,1.5)}},circle,fill,inner sep=2pt] at (axis cs:1,1.5){};
                    }
                \end{axis}
            \end{tikzpicture}
            \captionof*{figure}{$f(x)x^3+1,5$}
    \end{minipage}
    
\end{task}
\clearpage
% Aufgabe 7
\begin{task}[points=1]{weitere Anwendungen der 1. Ableitung}

\end{task}
% Aufgabe 8
\begin{task}[points=1]{Monotonieverhalten + rel. Extremwerte}

\end{task}
% Aufgabe 9
\begin{task}[points=1]{Evtl. Krümmungsverhalten und Wendepunkte}

\end{task}
% \section*{Stochastik}
% % Aufgabe 10
% \begin{task}[points=1]{relative und absolute Häufigkeiten berechnen}

% \end{task}
% % Aufgabe 11
% \begin{task}[points=1]{Häufigkeiten in eine Vierfeldertafel eintragen}

% \end{task}
% % Aufgabe 12
% \begin{task}[points=1]{Rechnen mit Wahrscheinlichkeiten}

% \end{task}
% % Aufgabe 13
% \begin{task}[points=1]{Baumdiagramm aufstellen}

% \end{task}
% % Aufgabe 14
% \begin{task}[points=1]{Pfadregeln anwenden}

% \end{task}
% % Aufgabe 15
% \begin{task}[points=1]{Auf stochastische Unabhängigkeit prüfen}

% \end{task}
% % Aufgabe 16
% \begin{task}[points=1]{bedingte Wahrscheinlichkeiten berechnen}

% \end{task}
% % Aufgabe 17
% \begin{task}[points=1]{Ergebnisse im Sachzusammenhang interpretieren}

% \end{task}
\end{document}
