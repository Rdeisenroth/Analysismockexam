\documentclass[
    ngerman,
    color=1b,
    load_common,
    % submission,
    leqno,
    % dark_mode,
    boxarc,
    % shell_escape=false,
    solution=true,
]{rubos-tuda-template}


% --Definitionen für das Dokument (alle Optional)--
% \sheetnumber{1}
\version{1.0 Snapshot}
\semester{Sommersemester 2021}
\topic{Analysis und Stochastik}
%  \fachbereich{Mathematik}
\author{Ruben Deisenroth\and Simon Hüner}
% \contributor{Simon Hüner}
\date{\today}

% --Stilistische Anpassungen für die Klausur--
\ExplSyntaxOn
% \renewcommand*{\printAuthor}[1]{
%   #1
% }
\tl_gset:Nn \g_ptxcd_logofile_tl {}

\ExplSyntaxOff
\renewcommand*{\taskformat}{\taskname\tasksep\thetask{}} % Taskprefix
\renewcommand*{\subtaskformat}{\thesubtask\enskip}
\renewcommand{\printTopic}{\fatsf{Themen:}~\getTopic{}}
% Titelbereich
\termStyle{left-right-manual}
\termLeft{printAuthor,printContributor,printTopic}
\termRight{printSemester,printVersion}
\term{\fatsf{Bearbeitungszeit:} 90 Minuten\\\printDate{}\hfill\printFachbereich{}}
% Kopfzeile
\ConfigureHeadline{
	headline={aud}
}

% Dokumentenmakros
\RequirePackage{pifont}
\newcommand{\tickedbox}{\huge\rlap{$\square$}{\Huge\hspace{1pt}\ding{55}}}
\newcommand{\redtickedbox}{\huge\rlap{$\square$}{\Huge\hspace{1pt}\color{red}\ding{55}}}
\newcommand{\untickedbox}{\huge\rlap{$\square$}}
% \pgfplotsset{compat=1.17}

% --Beginn Dokument--

\begin{document}
\title[Mathe Klausur]{Übungsklausur\\im Fach Mathematik}
\maketitle{}
\newcommand{\titledistance}{\\[1em]} % Distance for title text frame
\newcommand{\unchecked}{\rlap{\(\square\)}} % Checkbox
\begin{minipage}[t]{0.45 \textwidth}
    Name: \IfSolutionT{\rlap{\raisebox{2pt}{\fatsf{~Musterlösung}}}}\dotfill
    \titledistance
    Vorname: \IfSolutionT{\rlap{\raisebox{2pt}{\fatsf{~-}}}}\dotfill
    \titledistance
    Matrikelnummer: \IfSolutionT{\rlap{\raisebox{2pt}{\fatsf{~0000000}}}}\dotfill
    \titledistance
    Unterschrift: \IfSolutionT{\rlap{\raisebox{2pt}{\fatsf{-}}}}\dotfill
\end{minipage}
\hfill\vline\hfill
\begin{minipage}[t]{0.45 \textwidth}
    Studiengang: \IfSolutionT{\rlap{\raisebox{2pt}{\fatsf{~-}}}}\dotfill
    \titledistance
    Semester: \IfSolutionT{\rlap{\raisebox{2pt}{\fatsf{~-}}}}\dotfill
    \titledistance
    Anzahl Abgegebener Zusatzblätter:\IfSolutionT{\rlap{\raisebox{2pt}{\fatsf{~0}}}} \dotfill
    \titledistance
    Drittversuch? Bitte ggf. ankreuzen \hspace{0.5cm} \huge\unchecked\normalsize
\end{minipage}
% Punktetabelle

\begin{figure}[ht]
    \centering
    \begin{tikzpicture}
        \tikzstyle{gradetablenode}=[minimum size=1cm, draw]
        \tikzstyle{lightgraygradetablenode}=[gradetablenode, fill=fgcolor!10!\thepagecolor]
        \node[lightgraygradetablenode, text width=3cm, font=\sffamily, anchor=west] (n0-0) at (0, 1){Aufgabe};
        \node[gradetablenode, text width=3cm, font=\sffamily, below right = -\pgflinewidth and 0cm of n0-0.south west] (n1-0){Punkte (max)};
        \node[gradetablenode, text width=3cm, font=\sffamily, below right = -\pgflinewidth and 0cm of n1-0.south west] (n2-0){Punkte (erreicht)};
        \xdef\lasttask{0}
        \mapTasks{
            \node[lightgraygradetablenode, font=\bfseries, right=-\pgflinewidth of n0-\the\numexpr\value{task}-1\relax.east] (n0-\the\value{task}) {\thetask{}};
            \node[gradetablenode, right=-\pgflinewidth of n1-\the\numexpr\value{task}-1\relax.east] (n1-\the\value{task}) {\getPoints{\thetask}};
            \node[gradetablenode, right=-\pgflinewidth of n2-\the\numexpr\value{task}-1\relax.east] (n2-\the\value{task}) {\IfSolutionT{\fatsf{\getPoints{\thetask}}}};
            \xdef\lasttask{\numexpr\lasttask+1\relax}
        }
        % Probably + 1 in node names is too much, will check later
        \node[lightgraygradetablenode, minimum width=1.1cm, right=-\pgflinewidth of n0-\the\numexpr\lasttask\relax.east] (n0-\the\numexpr\lasttask + 1\relax){$\boldsymbol{\Sigma}$};
        \node[gradetablenode, minimum width=1.1cm, right=-\pgflinewidth of n1-\the\numexpr\lasttask\relax.east] (n1-\the\numexpr\lasttask + 1\relax){$\getPointsTotal{}$};
        \node[gradetablenode, minimum width=1.1cm, right=-\pgflinewidth of n2-\the\numexpr\lasttask\relax.east] (n2-\the\numexpr\lasttask + 1\relax){\IfSolutionT{\fatsf{\getPointsTotal{}}}};
        % % Extra lines
        \draw[transform canvas={xshift = -1mm}] (n0-0.north east) -- (n2-0.south east);
        \draw[transform canvas={xshift = 1mm}] (n0-\the\numexpr\lasttask+1.north west) -- (n2-\the\numexpr\lasttask+1.south west);
        \draw[thick] (n0-0.north west) rectangle (n2-\the\numexpr\lasttask+1.south east);
    \end{tikzpicture}
    % \caption*{Punkteübersicht}
\end{figure}
{\large \textbf{Hinweise:}}
\begin{itemize}
    \item Überprüfen Sie zunächst, ob Ihre Klausur die Seiten 1 bis \pageref*{LastPage} besitzt.
    \item Füllen Sie zuerst das Deckblatt aus und halten Sie Studienausweis und Lichtbildausweis bereit.
    \item Sie sollten die Lösungen möglichst direkt in die Klausur eintragen. Reicht der vorhandene Platz nicht aus, so können Sie zusätzliche Blätter verwenden, die Sie zuerst mit Nachname, Vorname und Matrikelnummer kennzeichnen.
    \item Nicht mit Bleistift schreiben und keine roten oder grünen Stifte verwenden.
    \item Erlaubte Hilfsmittel: \textbf{Taschenrechner}, \textbf{Formelsamlung}
\end{itemize}
{\large \textbf{Viel Erfolg!}}
\clearpage



%% --Beginn Hausübung--%%
\section*{Analysis (Differenzialrechnung)}
% Aufgabe 1
\begin{task}[points=8]{Differenzen- und Diffenrenzialquotient anwenden}
    \begin{grayInfoBox}
        Beantworten Sie die Fragen a) bis c) in \textbf{vollständigen} Sätzen und zeigen Sie in den Aufgabenteilen d) und e) Ihren \textbf{gesamten} Rechenweg.
    \end{grayInfoBox}
    \begin{cpenumerate}[label=\alph*), itemsep=1em]
        \item Wie ist der Differenzenquotient definiert?\\[2ex]
        \mbox{}
        \IfSolutionT{\rlap{\raisebox{2pt}{\textcolor{red}{~Der Differenzenquotient ist die mittlere Steigung zwischen zwei Punkten $(x_1,y_1)$ und $(x_2,y_2)$ einer Funktion}}}}
        \dotfill\\[2ex]
        \mbox{}
        \IfSolutionT{\rlap{\raisebox{2pt}{\textcolor{red}{~$f(x)$.}}}}
        \dotfill{}
        \item Was unterscheidet den Diffenrenzialquotienten vom Differenzenquotienten?\\[2ex]
        \mbox{}
        \IfSolutionT{\rlap{\raisebox{2pt}{\textcolor{red}{~Der Diffenrenzialquotienten ist der Differenzenquotient eines Punktes $P_1=(x_1,y_1)$ auf einer Funktion f(x)}}}}
        \dotfill\\[2ex]
        \mbox{}
        \IfSolutionT{\rlap{\raisebox{2pt}{\textcolor{red}{~dessen zweiter Punkt $(x_2,y_2)$ gegen diesen in­fi­ni­te­si­mal genau strebt. Daurch kann die Steigung in Punkt $P_1$}}}}
        \dotfill\\[2ex]
        \mbox{}
        \IfSolutionT{\rlap{\raisebox{2pt}{\textcolor{red}{~beschrieben werden.}}}}
        \dotfill\\[2ex]
        \mbox{}
        \dotfill\\[2ex]
        \mbox{}
        \dotfill
        \item Wann in welchen Fällen sind Differenzialquotient und Differenzenquotient identisch über die gesamte Funktion?\\[2em]
        \mbox{}
        \IfSolutionT{\rlap{\raisebox{2pt}{\textcolor{red}{Dieser Fall lieght vor wenn die Funktion $f(x)$ eine lineare Funktion ist und dadurch $f'(x)$ eine Konstante ist. So-}}}}
        \dotfill\\[2ex]
        \mbox{}
        \IfSolutionT{\rlap{\raisebox{2pt}{\textcolor{red}{mit ist die Steigung in jedem Punkt identisch und die mittlere Steigung zwischen zwei Punkt entspricht dieser.}}}}
        \dotfill\\[2ex]
        \mbox{}
        \dotfill\\[2ex]
        \mbox{}
        \dotfill
        \item Differenzenquotient von: $f(x)=2x^3$ mit $x\in\mathbb{R} [0;2]$\par
        \IfSolutionT{\vbox to 0pt{\hbox{\qquad\textcolor{red}{$\dfrac{\Delta y}{\Delta x} = \dfrac{f(x_2)-f(x_1)}{x_2-x_1} \\[0.2ex] \Rightarrow \dfrac{f(2)-f(0)}{2-0} =\dfrac{4-0}{2} = 2 $}}\vss}}
        \mbox{}\vspace{4em}
        \item Differenzialquotient von: $g(x)=x^2$ mit $x\to1$
        \IfSolutionT{\textcolor{red}{\\[0.2ex] $\dfrac{dx}{dy} = \lim_{x_2\to x_1} \dfrac{f(x_2)-f(x_1)}{x_2-x_1} = \lim_{h\to0} \dfrac{f(x_1+h)-f(x_1)}{h}
                \\[0.2ex]  \Rightarrow \lim_{h\to0} \dfrac{f(1+h)-f(1)}{h} = \dfrac{1+2h+h^2-1}{h} = \dfrac{2h+h^2}{h} = 2+h = 2$}}
    \end{cpenumerate}
\end{task}
\clearpage
% Aufgabe 2
\begin{task}[points=9]{Ganzrationale Funktionen ableiten (Potenz-, Faktor- und Summenregel)}
    \begin{grayInfoBox}
        Bilden Sie für jede der gegebene Funktion die zugehöhrige Ableitung.
    \end{grayInfoBox}
    \begin{cpenumerate}[label=\alph*), itemsep=1em]
        \item $f(x)=3x^4-53$
        \item $g(x)=0,2x^5-2,5x^4+3x^2-2,79x+23$
        \item $h(t)=0,25x^4-0,5t^2+x^2-6t+5x-3$
        \item $i(x)=\dfrac{2x^6-8x^4}{x^3+2x^2}$
        \item $j(u)=\dfrac{5u^6+au^4-bu^2}{2au^2*(3a-5b)}$
        \item $k(l)=5l^{2,5}+l^2-2l^{-3,5}$
    \end{cpenumerate}
    \begin{tikzpicture}[x=1cm, y=1cm]
        % \draw[step=1mm, line width=0.1mm, black!30!white] (0,0) grid (\textwidth,-15cm);
        \draw[step=5mm, black!40!white] (0,0) grid (\textwidth,-15cm);
        % \draw[step=5cm, line width=0.5mm, black!50!white] (0,0) grid (\textwidth,-15cm);
        \draw[step=1cm, line width=.8pt, black!90!white] (0,0) grid (\textwidth,-15cm);
        \IfSolutionT{
            \node[anchor= north west] at (0,0){%
                \textcolor{red}{%
                    $\begin{aligned}
                        f'(x) & =12x^3                       \\
                        g'(x) & =x^4-10x^3+6x-2,79           \\
                        h'(t) & =-t-6                        \\
                        i'(x) & =6x^2-8x                     \\
                        j'(u) & =(6a^2-5ab)^{-1}*(20u^3+2au) \\
                        k'(l) & =12,5l^{1,5}+2l+7l^{-4,5}
                    \end{aligned}$
                    }
            };
        }
    \end{tikzpicture}
\end{task}
\clearpage
% Aufgabe 3
\begin{task}[points=6]{Zusammenhang zwischen Funktion und Ableitungsfunktion begründen}
    % Koordinatensystemgröße für Gesamte Aufgabe
    \def\coordinatesystemradius{4}

    % Originalgröße
    % \begin{tikzpicture}
    %     \begin{axis}[
    %         xlabel=$x$,
    %         ylabel={$y$},
    %         axis lines=middle,
    %         axis line style={-{Triangle}{Bar}, thick},
    %         xmin=-\the\dimexpr\coordinatesystemradius pt + .5pt\relax,
    %         xmax=\the\dimexpr\coordinatesystemradius pt + .5pt\relax,
    %         ymin=-\the\dimexpr\coordinatesystemradius pt + .5pt\relax,
    %         ymax=\the\dimexpr\coordinatesystemradius pt + .5pt\relax,
    %         x=1cm,
    %         y=1cm,
    %         grid=both,
    %         minor tick num=1,
    %         xtick={-\coordinatesystemradius,...,\coordinatesystemradius},
    %         ytick={-\coordinatesystemradius,...,\coordinatesystemradius},
    %         every tick/.style={color=fgcolor},
    %         major grid style = {line width=.8pt},
    %         ]
    %         \addplot[smooth, draw=accentcolor,very thick]{x^2-2};
    %     \end{axis}
    % \end{tikzpicture}

    %50%
    % \begin{tikzpicture}
    %     \def\coordinatesystemsize{\coordinatesystemradius}
    %     % Odd System Size
    %     \ifthenelse{\isodd{\coordinatesystemsize}}{}{
    %         \def\coordinatesystemsize{\the\numexpr\coordinatesystemradius + 1\relax}
    %     }
    %     \begin{axis}[
    %         xlabel=$x$,
    %         ylabel={$y$},
    %         axis lines=middle,
    %         axis line style={-{Triangle}{Bar}, thick},
    %         xmin=-\coordinatesystemsize,
    %         xmax=\coordinatesystemsize,
    %         ymin=-\coordinatesystemsize,
    %         ymax=\coordinatesystemsize,
    %         x=.5cm,
    %         y=.5cm,
    %         grid=both,
    %         minor tick num=1,
    %         xtick distance={2},
    %         ytick distance={2},
    %         every tick/.style={color=fgcolor},
    %         major grid style = {line width=.8pt},
    %         ]
    %         \addplot[smooth, draw=accentcolor,very thick]{x^2-2};
    %     \end{axis}
    % \end{tikzpicture}

    \begin{grayInfoBox}
        Kreuzen sie bei den Graphen $f(x)$, $g(x)$ und $h(x)$ jeweils die richtige Ableitung an. \textbf{Begründen} Sie Ihre Antwort
    \end{grayInfoBox}
    \begin{figure}[ht]
        \centering
        %f(x)
        \begin{subfigure}[t]{.24\textwidth}
            \centering
            \begin{tikzpicture}
                \def\coordinatesystemsize{\coordinatesystemradius}
                % Odd System Size
                % \ifthenelse{\isodd{\coordinatesystemsize}}{}{
                %     \def\coordinatesystemsize{\the\numexpr\coordinatesystemradius + 1\relax}
                % }
                \begin{axis}[
                    xlabel=$x$,
                    ylabel={$y$},
                    axis lines=middle,
                    axis line style={-{Triangle}, thick},
                    xmin=-\coordinatesystemsize,
                    xmax=\coordinatesystemsize,
                    ymin=-\coordinatesystemsize,
                    ymax=\coordinatesystemsize,
                    x=.5cm,
                    y=.5cm,
                    grid=both,
                    minor tick num=1,
                    xtick distance={2},
                    ytick distance={2},
                    every tick/.style={color=fgcolor, thick},
                    major grid style = {line width=.8pt},
                    ]
                    \addplot[smooth, samples=100, draw=accentcolor,very thick]{x^2-2};
                \end{axis}
            \end{tikzpicture}
            \caption*{$f(x)$}
        \end{subfigure}
        \begin{subfigure}[t]{.25\textwidth}
            \centering
            \begin{tikzpicture}
                \def\coordinatesystemsize{\coordinatesystemradius}
                % Odd System Size
                % \ifthenelse{\isodd{\coordinatesystemsize}}{}{
                %     \def\coordinatesystemsize{\the\numexpr\coordinatesystemradius + 1\relax}
                % }
                \begin{axis}[
                    xlabel=$x$,
                    ylabel={$y$},
                    axis lines=middle,
                    axis line style={-{Triangle}, thick},
                    xmin=-\coordinatesystemsize,
                    xmax=\coordinatesystemsize,
                    ymin=-\coordinatesystemsize,
                    ymax=\coordinatesystemsize,
                    x=.5cm,
                    y=.5cm,
                    grid=both,
                    minor tick num=1,
                    xtick distance={2},
                    ytick distance={2},
                    every tick/.style={color=fgcolor, thick},
                    major grid style = {line width=.8pt},
                    ]
                    \addplot[smooth, samples=100, draw=red,very thick]{x^2};
                \end{axis}
            \end{tikzpicture}
            \caption*{\untickedbox}
        \end{subfigure}
        \begin{subfigure}[t]{.25\textwidth}
            \centering
            \begin{tikzpicture}
                \def\coordinatesystemsize{\coordinatesystemradius}
                % Odd System Size
                % \ifthenelse{\isodd{\coordinatesystemsize}}{}{
                %     \def\coordinatesystemsize{\the\numexpr\coordinatesystemradius + 1\relax}
                % }
                \begin{axis}[
                    xlabel=$x$,
                    ylabel={$y$},
                    axis lines=middle,
                    axis line style={-{Triangle}, thick},
                    xmin=-\coordinatesystemsize,
                    xmax=\coordinatesystemsize,
                    ymin=-\coordinatesystemsize,
                    ymax=\coordinatesystemsize,
                    x=.5cm,
                    y=.5cm,
                    grid=both,
                    minor tick num=1,
                    xtick distance={2},
                    ytick distance={2},
                    every tick/.style={color=fgcolor, thick},
                    major grid style = {line width=.8pt},
                    ]
                    \addplot[smooth, samples=100, draw=red,very thick]{2*x};
                \end{axis}
            \end{tikzpicture}
            \caption*{\IfSolutionTF{\redtickedbox}{\untickedbox}}
        \end{subfigure}
        \begin{subfigure}[t]{.24\textwidth}
            \centering
            \begin{tikzpicture}
                \def\coordinatesystemsize{\coordinatesystemradius}
                % Odd System Size
                % \ifthenelse{\isodd{\coordinatesystemsize}}{}{
                %     \def\coordinatesystemsize{\the\numexpr\coordinatesystemradius + 1\relax}
                % }
                \begin{axis}[
                    xlabel=$x$,
                    ylabel={$y$},
                    axis lines=middle,
                    axis line style={-{Triangle}, thick},
                    xmin=-\coordinatesystemsize,
                    xmax=\coordinatesystemsize,
                    ymin=-\coordinatesystemsize,
                    ymax=\coordinatesystemsize,
                    x=.5cm,
                    y=.5cm,
                    grid=both,
                    minor tick num=1,
                    xtick distance={2},
                    ytick distance={2},
                    every tick/.style={color=fgcolor, thick},
                    major grid style = {line width=.8pt},
                    ]
                    \addplot[smooth, samples=100, draw=red,very thick]{-2*x};
                \end{axis}
            \end{tikzpicture}
            \caption*{\untickedbox}
        \end{subfigure}
        %g(x)
        \begin{subfigure}[t]{.24\textwidth}
            \centering
            \begin{tikzpicture}
                \def\coordinatesystemsize{\coordinatesystemradius}
                % Odd System Size
                % \ifthenelse{\isodd{\coordinatesystemsize}}{}{
                %     \def\coordinatesystemsize{\the\numexpr\coordinatesystemradius + 1\relax}
                % }
                \begin{axis}[
                    xlabel=$x$,
                    ylabel={$y$},
                    axis lines=middle,
                    axis line style={-{Triangle}, thick},
                    xmin=-\coordinatesystemsize,
                    xmax=\coordinatesystemsize,
                    ymin=-\coordinatesystemsize,
                    ymax=\coordinatesystemsize,
                    x=.5cm,
                    y=.5cm,
                    grid=both,
                    minor tick num=1,
                    xtick distance={2},
                    ytick distance={2},
                    every tick/.style={color=fgcolor, thick},
                    major grid style = {line width=.8pt},
                    ]
                    \addplot[smooth, samples=100, draw=accentcolor,very thick]{(x-1)^3+2};
                \end{axis}
            \end{tikzpicture}
            \caption*{$g(x)$}
        \end{subfigure}
        \begin{subfigure}[t]{.25\textwidth}
            \centering
            \begin{tikzpicture}
                \def\coordinatesystemsize{\coordinatesystemradius}
                % Odd System Size
                % \ifthenelse{\isodd{\coordinatesystemsize}}{}{
                %     \def\coordinatesystemsize{\the\numexpr\coordinatesystemradius + 1\relax}
                % }
                \begin{axis}[
                    xlabel=$x$,
                    ylabel={$y$},
                    axis lines=middle,
                    axis line style={-{Triangle}, thick},
                    xmin=-\coordinatesystemsize,
                    xmax=\coordinatesystemsize,
                    ymin=-\coordinatesystemsize,
                    ymax=\coordinatesystemsize,
                    x=.5cm,
                    y=.5cm,
                    grid=both,
                    minor tick num=1,
                    xtick distance={2},
                    ytick distance={2},
                    every tick/.style={color=fgcolor, thick},
                    major grid style = {line width=.8pt},
                    ]
                    \addplot[smooth, samples=100, draw=red,very thick]{3*(x-1)^2};
                \end{axis}
            \end{tikzpicture}
            \caption*{\IfSolutionTF{\redtickedbox}{\untickedbox}}
        \end{subfigure}
        \begin{subfigure}[t]{.25\textwidth}
            \centering
            \begin{tikzpicture}
                \def\coordinatesystemsize{\coordinatesystemradius}
                % Odd System Size
                % \ifthenelse{\isodd{\coordinatesystemsize}}{}{
                %     \def\coordinatesystemsize{\the\numexpr\coordinatesystemradius + 1\relax}
                % }
                \begin{axis}[
                    xlabel=$x$,
                    ylabel={$y$},
                    axis lines=middle,
                    axis line style={-{Triangle}, thick},
                    xmin=-\coordinatesystemsize,
                    xmax=\coordinatesystemsize,
                    ymin=-\coordinatesystemsize,
                    ymax=\coordinatesystemsize,
                    x=.5cm,
                    y=.5cm,
                    grid=both,
                    minor tick num=1,
                    xtick distance={2},
                    ytick distance={2},
                    every tick/.style={color=fgcolor, thick},
                    major grid style = {line width=.8pt},
                    ]
                    \addplot[smooth, samples=100, draw=red,very thick]{2*(x-1)};
                \end{axis}
            \end{tikzpicture}
            \caption*{\untickedbox}
        \end{subfigure}
        \begin{subfigure}[t]{.24\textwidth}
            \centering
            \begin{tikzpicture}
                \def\coordinatesystemsize{\coordinatesystemradius}
                % Odd System Size
                % \ifthenelse{\isodd{\coordinatesystemsize}}{}{
                %     \def\coordinatesystemsize{\the\numexpr\coordinatesystemradius + 1\relax}
                % }
                \begin{axis}[
                    xlabel=$x$,
                    ylabel={$y$},
                    axis lines=middle,
                    axis line style={-{Triangle}, thick},
                    xmin=-\coordinatesystemsize,
                    xmax=\coordinatesystemsize,
                    ymin=-\coordinatesystemsize,
                    ymax=\coordinatesystemsize,
                    x=.5cm,
                    y=.5cm,
                    grid=both,
                    minor tick num=1,
                    xtick distance={2},
                    ytick distance={2},
                    every tick/.style={color=fgcolor, thick},
                    major grid style = {line width=.8pt},
                    ]
                    \addplot[smooth, samples=100, draw=red,very thick]{-3*(x-1)^2+2};
                \end{axis}
            \end{tikzpicture}
            \caption*{\untickedbox}
        \end{subfigure}
        %h(x)
        \begin{subfigure}[t]{.24\textwidth}
            \centering
            \begin{tikzpicture}
                \def\coordinatesystemsize{\coordinatesystemradius}
                % Odd System Size
                % \ifthenelse{\isodd{\coordinatesystemsize}}{}{
                %     \def\coordinatesystemsize{\the\numexpr\coordinatesystemradius + 1\relax}
                % }
                \begin{axis}[
                    xlabel=$x$,
                    ylabel={$y$},
                    axis lines=middle,
                    axis line style={-{Triangle}, thick},
                    xmin=-\coordinatesystemsize,
                    xmax=\coordinatesystemsize,
                    ymin=-\coordinatesystemsize,
                    ymax=\coordinatesystemsize,
                    x=.5cm,
                    y=.5cm,
                    % restrict y to domain=-100:100,
                    grid=both,
                    minor tick num=1,
                    xtick distance={2},
                    ytick distance={2},
                    every tick/.style={color=fgcolor, thick},
                    major grid style = {line width=.8pt},
                    ]
                    \addplot[smooth, samples=100, samples=100, draw=accentcolor,very thick]{.2 * (x)^2 * (x-2) * (x+2) - 2};
                \end{axis}
            \end{tikzpicture}
            \caption*{$h(x)$}
        \end{subfigure}
        \begin{subfigure}[t]{.25\textwidth}
            \centering
            \begin{tikzpicture}
                \def\coordinatesystemsize{\coordinatesystemradius}
                % Odd System Size
                % \ifthenelse{\isodd{\coordinatesystemsize}}{}{
                %     \def\coordinatesystemsize{\the\numexpr\coordinatesystemradius + 1\relax}
                % }
                \begin{axis}[
                    xlabel=$x$,
                    ylabel={$y$},
                    axis lines=middle,
                    axis line style={-{Triangle}, thick},
                    xmin=-\coordinatesystemsize,
                    xmax=\coordinatesystemsize,
                    ymin=-\coordinatesystemsize,
                    ymax=\coordinatesystemsize,
                    x=.5cm,
                    y=.5cm,
                    grid=both,
                    minor tick num=1,
                    xtick distance={2},
                    ytick distance={2},
                    every tick/.style={color=fgcolor, thick},
                    major grid style = {line width=.8pt},
                    ]
                    \addplot[smooth, samples=100, draw=red,very thick]{x^3};
                \end{axis}
            \end{tikzpicture}
            \caption*{\untickedbox}
        \end{subfigure}
        \begin{subfigure}[t]{.25\textwidth}
            \centering
            \begin{tikzpicture}
                \def\coordinatesystemsize{\coordinatesystemradius}
                % Odd System Size
                % \ifthenelse{\isodd{\coordinatesystemsize}}{}{
                %     \def\coordinatesystemsize{\the\numexpr\coordinatesystemradius + 1\relax}
                % }
                \begin{axis}[
                    xlabel=$x$,
                    ylabel={$y$},
                    axis lines=middle,
                    axis line style={-{Triangle}, thick},
                    xmin=-\coordinatesystemsize,
                    xmax=\coordinatesystemsize,
                    ymin=-\coordinatesystemsize,
                    ymax=\coordinatesystemsize,
                    x=.5cm,
                    y=.5cm,
                    grid=both,
                    minor tick num=1,
                    xtick distance={2},
                    ytick distance={2},
                    every tick/.style={color=fgcolor, thick},
                    major grid style = {line width=.8pt},
                    ]
                    \addplot[smooth, samples=100, draw=red,very thick]{-(0.8 * x^3 - 1.6 * x)};
                \end{axis}
            \end{tikzpicture}
            \caption*{\untickedbox}
        \end{subfigure}
        \begin{subfigure}[t]{.24\textwidth}
            \centering
            \begin{tikzpicture}
                \def\coordinatesystemsize{\coordinatesystemradius}
                % Odd System Size
                % \ifthenelse{\isodd{\coordinatesystemsize}}{}{
                %     \def\coordinatesystemsize{\the\numexpr\coordinatesystemradius + 1\relax}
                % }
                \begin{axis}[
                    xlabel=$x$,
                    ylabel={$y$},
                    axis lines=middle,
                    axis line style={-{Triangle}, thick},
                    xmin=-\coordinatesystemsize,
                    xmax=\coordinatesystemsize,
                    ymin=-\coordinatesystemsize,
                    ymax=\coordinatesystemsize,
                    x=.5cm,
                    y=.5cm,
                    grid=both,
                    minor tick num=1,
                    xtick distance={2},
                    ytick distance={2},
                    every tick/.style={color=fgcolor, thick},
                    major grid style = {line width=.8pt},
                    ]
                    \addplot[smooth, samples=100, draw=red,very thick]{0.8 * x^3 - 1.6 * x};
                \end{axis}
            \end{tikzpicture}
            \caption*{\IfSolutionTF{\redtickedbox}{\untickedbox}}
        \end{subfigure}
    \end{figure}
    \vspace{-1em}
    $f(x)$:~
    \IfSolutionT{\rlap{\raisebox{2pt}{\color{red}~Ableitung 1 ist stets positiv, aber $G_f$ fällt für $x<0$. Ableitung 3 ist für $x<0$ positiv, aber $G_f$ fällt für $x<0$.}}}
    \dotfill\\[1em]\mbox{}\dotfill\\[1em]
    $g(x)$:~
    \IfSolutionT{\rlap{\raisebox{2pt}{\color{red}~Ableitung $2$ und $3$ haben negative $y$ werte, aber $G_g$ ist monoton steigend.}}}
    \dotfill\\[1em]\mbox{}\dotfill\\[1em]
    $h(x)$:~
    \IfSolutionT{\rlap{\raisebox{2pt}{\color{red}~$g_h$ hat ein lokales Maximum bei $x=0$. Also muss bei $x=0$ ein VZW der Ableitung von \enquote{$+$} nach \enquote{$-$} stattfinden.}}}
    \dotfill\\[1em]\mbox{}\dotfill
\end{task}
\clearpage
% Aufgabe 4
\begin{task}[points=auto]{Graph der Ableitungsfunktion einer gegebenen Funktion zeichnen}
    \begin{grayInfoBox}
        Zeichnen sie in die gegbenen Koordinatensysteme jeweils den Graphen der Ableitungsfunktion $f'(x)$, welche zur dort bereits eingezeichneten Graphen der Funktion $f(x)$ gehöhrt.
        Achten Sie dabei inbesondere auf die charakteristischen Punkte und eine saubere Stiftführung. Den weißen Bereich am rechten Seitenrand können sie für Notizen und Überlegungen nutzen.
    \end{grayInfoBox}
    \par
    \begin{subtask}[points=1]
        \begin{tikzpicture}
            \begin{axis}[
                xmin=-5.2,
                ymin=-5.2,
                xmax=5.4,
                ymax=5.4,
                width=10.6cm,
                height=10.6cm,
                scale only axis,
                grid=both,
                axis x line=middle,
                axis y line=center,
                minor tick num=1,
                ticklabel style = {font=\footnotesize},
                y axis line style ={-{Triangle[angle=45:0.4em]}},
                xtick={-5,...,5},
                ytick={-5,...,5},
                x axis line style ={-{Triangle[angle=45:0.4em]}},
                axis line style={thick},
                tick style = {thick, fgcolor},
                major grid style = {line width=.8pt},
                ylabel={$y$},
                xlabel={$x$}
                ]
                \addplot[thick,samples=100, smooth, blue]{0.75*x^2-2.25*x-2};%
                \IfSolutionT{
                    \addplot[thick,samples=100, smooth, red]{1.5*x-2.25};
                    \node[font=\small, red]at(axis cs:2,1.7){$f'(x)$};
                }
                \node[font=\small, blue]at(axis cs:-1.8,2.2){$f(x)$};
            \end{axis}
        \end{tikzpicture}
    \end{subtask}
    \begin{subtask}[points=4]
        \begin{tikzpicture}
            \begin{axis}[
                xmin=-5.2,
                ymin=-5.2,
                xmax=5.4,
                ymax=5.4,
                width=10.6cm,
                height=10.6cm,
                scale only axis,
                grid=both,
                axis x line=middle,
                axis y line=center,
                minor tick num=1,
                ticklabel style = {font=\footnotesize},
                y axis line style ={-{Triangle[angle=45:0.4em]}},
                xtick={-5,...,5},
                ytick={-5,...,5},
                x axis line style ={-{Triangle[angle=45:0.4em]}},
                axis line style={thick},
                tick style = {thick, fgcolor},
                major grid style = {line width=.8pt},
                ylabel={$y$},
                xlabel={$x$}
                ]
                \addplot[thick,samples=100, smooth, green!50!black]{(0.05*x^4+(1/15)*x^3-0.6*x^2)-1};%
                \IfSolutionT{
                    \addplot[thick,samples=100, smooth, red]{(1/5)*x*(x^2+x-6)};%
                    \node[font=\small, red]at(axis cs:2,1.7){$f'(x)$};
                }
                \node[font=\small, green!50!black]at(axis cs:3,-1.7){$f(x)$};
            \end{axis}
        \end{tikzpicture}
    \end{subtask}
    \begin{subtask}[points=2]
        \begin{tikzpicture}
            \begin{axis}[
                xmin=-5.2,
                ymin=-3.2,
                xmax=5.4,
                ymax=3.4,
                width=10.6cm,
                height=6.6cm,
                scale only axis,
                grid=both,
                axis x line=middle,
                axis y line=center,
                minor tick num=1,
                ticklabel style = {font=\footnotesize},
                y axis line style ={-{Triangle[angle=45:0.4em]}},
                xticklabels={
                        $-2\pi$, $-\frac{3\pi}{2}$, $-\pi$, $-\frac{\pi}{2}$, 0,
                        $\frac{\pi}{2}$, $\pi$, $\frac{3\pi}{2}$, $2\pi$
                    },
                xtick={
                        -6.28318,-4.71238898,..., 6.28318
                    },
                ytick={-5,...,5},
                x axis line style ={-{Triangle[angle=45:0.4em]}},
                axis line style={thick},
                tick style = {thick, black},
                major grid style = {line width=.8pt},
                ylabel={$y$},
                xlabel={$x$}
                ]
                \addplot[thick,samples=100, smooth, orange!50!black]{-3*sin(deg(x))};%
                \IfSolutionT{
                    \addplot[thick,samples=100, smooth, red]{-3*cos(deg(x))};%
                    \node[font=\small, red]at(axis cs:1.8,2){$f'(x)$};
                }
                \node[font=\small, orange!50!black]at(axis cs:-3,2){$f(x)$};
            \end{axis}
        \end{tikzpicture}
    \end{subtask}
\end{task}
\clearpage
% Aufgabe 5
\begin{task}[points=3]{Funktionsterme und Funktionsgraphen zuordnen}
    \begin{grayInfoBox}
        Gegeben sind die folgenden Funktionen:\begin{align*}
            f(x) & =3\cdot(x-1)^2                            \\
            g(x) & =2\cdot(x-1)+0.25                         \\
            h(x) & =2\cdot(x+1)-4                            \\
            i(x) & =-3\cdot(x-1)^2+2                         \\
            j(x) & =(x+2)^3+2                                \\
            k(x) & =2\cdot\sin(x)                            \\
            l(x) & =-0.2 (x)^2 \cdot (x-2) \cdot (x+2) + 1.4 \\
            m(x) & =.2 (x)^2 \cdot (x-2) \cdot (x+2) - 2     \\
        \end{align*}
        Ihre Grafen werden mit $G_f$, $G_g$ usw. bezeichnet und sind im Folgenden Abgebildet. Schreiben Sie unter jedes Bild, welchen Graphen es darstellt. Es ist \textbf{keine} Begründung notwendig.
    \end{grayInfoBox}
    \def\coordinatesystemradius{4}
    \begin{figure}[ht]
        \centering
        %f(x)
        \begin{subfigure}[t]{.24\textwidth}
            \centering
            \begin{tikzpicture}
                \def\coordinatesystemsize{\coordinatesystemradius}
                Odd System Size
                % \ifthenelse{\isodd{\coordinatesystemsize}}{}{
                %     \def\coordinatesystemsize{\the\numexpr\coordinatesystemradius + 1\relax}
                % }
                \begin{axis}[
                    xlabel=$x$,
                    ylabel={$y$},
                    axis lines=middle,
                    axis line style={-{Triangle}, thick},
                    xmin=-\coordinatesystemsize,
                    xmax=\coordinatesystemsize,
                    ymin=-\coordinatesystemsize,
                    ymax=\coordinatesystemsize,
                    x=.5cm,
                    y=.5cm,
                    grid=both,
                    minor tick num=1,
                    xtick distance={2},
                    ytick distance={2},
                    every tick/.style={color=fgcolor, thick},
                    major grid style = {line width=.8pt},
                    ]
                    \addplot[smooth, samples=100, draw=accentcolor,very thick]{(x+2)^3+2};
                \end{axis}
            \end{tikzpicture}
            \caption*{\tikz{\node[inner sep=0pt,draw, minimum width = 1.5cm, minimum height=.8cm]{\IfSolutionT{\color{red}$G_j$}};}}
        \end{subfigure}
        \begin{subfigure}[t]{.25\textwidth}
            \centering
            \begin{tikzpicture}
                \def\coordinatesystemsize{\coordinatesystemradius}
                % Odd System Size
                % \ifthenelse{\isodd{\coordinatesystemsize}}{}{
                %     \def\coordinatesystemsize{\the\numexpr\coordinatesystemradius + 1\relax}
                % }
                \begin{axis}[
                    xlabel=$x$,
                    ylabel={$y$},
                    axis lines=middle,
                    axis line style={-{Triangle}, thick},
                    xmin=-\coordinatesystemsize,
                    xmax=\coordinatesystemsize,
                    ymin=-\coordinatesystemsize,
                    ymax=\coordinatesystemsize,
                    x=.5cm,
                    y=.5cm,
                    grid=both,
                    minor tick num=1,
                    xtick distance={2},
                    ytick distance={2},
                    every tick/.style={color=fgcolor, thick},
                    major grid style = {line width=.8pt},
                    ]
                    \addplot[smooth, samples=100, draw=teal,very thick]{.2 * (x)^2 * (x-2) * (x+2) - 2};
                \end{axis}
            \end{tikzpicture}
            \caption*{\tikz{\node[inner sep=0pt,draw, minimum width = 1.5cm, minimum height=.8cm]{\IfSolutionT{\color{red}$G_m$}};}}
        \end{subfigure}
        \begin{subfigure}[t]{.25\textwidth}
            \centering
            \begin{tikzpicture}
                \def\coordinatesystemsize{\coordinatesystemradius}
                % Odd System Size
                % \ifthenelse{\isodd{\coordinatesystemsize}}{}{
                %     \def\coordinatesystemsize{\the\numexpr\coordinatesystemradius + 1\relax}
                % }
                \begin{axis}[
                    xlabel=$x$,
                    ylabel={$y$},
                    axis lines=middle,
                    axis line style={-{Triangle}, thick},
                    xmin=-\coordinatesystemsize,
                    xmax=\coordinatesystemsize,
                    ymin=-\coordinatesystemsize,
                    ymax=\coordinatesystemsize,
                    x=.5cm,
                    y=.5cm,
                    grid=both,
                    minor tick num=1,
                    xtick distance={2},
                    ytick distance={2},
                    every tick/.style={color=fgcolor, thick},
                    major grid style = {line width=.8pt},
                    ]
                    \addplot[smooth, samples=100, draw=magenta,very thick]{2*(x-1)+0.25};
                \end{axis}
            \end{tikzpicture}
            \caption*{\tikz{\node[inner sep=0pt,draw, minimum width = 1.5cm, minimum height=.8cm]{\IfSolutionT{\color{red}$G_g$}};}}
        \end{subfigure}
        \begin{subfigure}[t]{.24\textwidth}
            \centering
            \begin{tikzpicture}
                \def\coordinatesystemsize{\coordinatesystemradius}
                % Odd System Size
                % \ifthenelse{\isodd{\coordinatesystemsize}}{}{
                %     \def\coordinatesystemsize{\the\numexpr\coordinatesystemradius + 1\relax}
                % }
                \begin{axis}[
                    xlabel=$x$,
                    ylabel={$y$},
                    axis lines=middle,
                    axis line style={-{Triangle}, thick},
                    xmin=-\coordinatesystemsize,
                    xmax=\coordinatesystemsize,
                    ymin=-\coordinatesystemsize,
                    ymax=\coordinatesystemsize,
                    x=.5cm,
                    y=.5cm,
                    grid=both,
                    minor tick num=1,
                    xtick distance={2},
                    ytick distance={2},
                    every tick/.style={color=fgcolor, thick},
                    major grid style = {line width=.8pt},
                    ]
                    \addplot[smooth, samples=100, draw=green!50!fgcolor,very thick]{2*sin(deg(x))};
                \end{axis}
            \end{tikzpicture}
            \caption*{\tikz{\node[inner sep=0pt,draw, minimum width = 1.5cm, minimum height=.8cm]{\IfSolutionT{\color{red}$G_k$}};}}
        \end{subfigure}
        %g(x)
        \begin{subfigure}[t]{.24\textwidth}
            \centering
            \begin{tikzpicture}
                \def\coordinatesystemsize{\coordinatesystemradius}
                % Odd System Size
                % \ifthenelse{\isodd{\coordinatesystemsize}}{}{
                %     \def\coordinatesystemsize{\the\numexpr\coordinatesystemradius + 1\relax}
                % }
                \begin{axis}[
                    xlabel=$x$,
                    ylabel={$y$},
                    axis lines=middle,
                    axis line style={-{Triangle}, thick},
                    xmin=-\coordinatesystemsize,
                    xmax=\coordinatesystemsize,
                    ymin=-\coordinatesystemsize,
                    ymax=\coordinatesystemsize,
                    x=.5cm,
                    y=.5cm,
                    grid=both,
                    minor tick num=1,
                    xtick distance={2},
                    ytick distance={2},
                    every tick/.style={color=fgcolor, thick},
                    major grid style = {line width=.8pt},
                    ]
                    \addplot[smooth, samples=100, draw=orange,very thick]{-.2 * (x)^2 * (x-2) * (x+2) + 1.4};
                \end{axis}
            \end{tikzpicture}
            \caption*{\tikz{\node[inner sep=0pt,draw, minimum width = 1.5cm, minimum height=.8cm]{\IfSolutionT{\color{red}$G_l$}};}}
        \end{subfigure}
        \begin{subfigure}[t]{.25\textwidth}
            \centering
            \begin{tikzpicture}
                \def\coordinatesystemsize{\coordinatesystemradius}
                % Odd System Size
                % \ifthenelse{\isodd{\coordinatesystemsize}}{}{
                %     \def\coordinatesystemsize{\the\numexpr\coordinatesystemradius + 1\relax}
                % }
                \begin{axis}[
                    xlabel=$x$,
                    ylabel={$y$},
                    axis lines=middle,
                    axis line style={-{Triangle}, thick},
                    xmin=-\coordinatesystemsize,
                    xmax=\coordinatesystemsize,
                    ymin=-\coordinatesystemsize,
                    ymax=\coordinatesystemsize,
                    x=.5cm,
                    y=.5cm,
                    grid=both,
                    minor tick num=1,
                    xtick distance={2},
                    ytick distance={2},
                    every tick/.style={color=fgcolor, thick},
                    major grid style = {line width=.8pt},
                    ]
                    \addplot[smooth, samples=100, draw=brown,very thick]{3*(x-1)^2};
                \end{axis}
            \end{tikzpicture}
            \caption*{\tikz{\node[inner sep=0pt,draw, minimum width = 1.5cm, minimum height=.8cm]{\IfSolutionT{\color{red}$G_f$}};}}
        \end{subfigure}
        \begin{subfigure}[t]{.25\textwidth}
            \centering
            \begin{tikzpicture}
                \def\coordinatesystemsize{\coordinatesystemradius}
                % Odd System Size
                % \ifthenelse{\isodd{\coordinatesystemsize}}{}{
                %     \def\coordinatesystemsize{\the\numexpr\coordinatesystemradius + 1\relax}
                % }
                \begin{axis}[
                    xlabel=$x$,
                    ylabel={$y$},
                    axis lines=middle,
                    axis line style={-{Triangle}, thick},
                    xmin=-\coordinatesystemsize,
                    xmax=\coordinatesystemsize,
                    ymin=-\coordinatesystemsize,
                    ymax=\coordinatesystemsize,
                    x=.5cm,
                    y=.5cm,
                    grid=both,
                    minor tick num=1,
                    xtick distance={2},
                    ytick distance={2},
                    every tick/.style={color=fgcolor, thick},
                    major grid style = {line width=.8pt},
                    ]
                    \addplot[smooth, samples=100, draw=olive,very thick]{2*(x-1)};
                \end{axis}
            \end{tikzpicture}
            \caption*{\tikz{\node[inner sep=0pt,draw, minimum width = 1.5cm, minimum height=.8cm]{\IfSolutionT{\color{red}$G_h$}};}}
        \end{subfigure}
        \begin{subfigure}[t]{.24\textwidth}
            \centering
            \begin{tikzpicture}
                \def\coordinatesystemsize{\coordinatesystemradius}
                % Odd System Size
                % \ifthenelse{\isodd{\coordinatesystemsize}}{}{
                %     \def\coordinatesystemsize{\the\numexpr\coordinatesystemradius + 1\relax}
                % }
                \begin{axis}[
                    xlabel=$x$,
                    ylabel={$y$},
                    axis lines=middle,
                    axis line style={-{Triangle}, thick},
                    xmin=-\coordinatesystemsize,
                    xmax=\coordinatesystemsize,
                    ymin=-\coordinatesystemsize,
                    ymax=\coordinatesystemsize,
                    x=.5cm,
                    y=.5cm,
                    grid=both,
                    minor tick num=1,
                    xtick distance={2},
                    ytick distance={2},
                    every tick/.style={color=fgcolor, thick},
                    major grid style = {line width=.8pt},
                    ]
                    \addplot[smooth, samples=100, draw=red,very thick]{-3*(x-1)^2+2};
                \end{axis}
            \end{tikzpicture}
            \caption*{\tikz{\node[inner sep=0pt,draw, minimum width = 1.5cm, minimum height=.8cm]{\IfSolutionT{\color{red}$G_i$}};}}
        \end{subfigure}
    \end{figure}
    Platz für Überlegungen:
\end{task}
\clearpage
% Aufgabe 6
\begin{task}[points=5]{Aufstellen von Tangenten und Normalen}
    \def\coordinatesystemradius{4}
    \begin{grayInfoBox}
        Gegeben sei der Graph $f(x):=x^3+0,5$.
        \textbf{Berechnen} Sie die Tangente $t_f$ bei $x=1$.
        \textbf{Geben} Sie auch den Schnittpunkt $p$ der Tangente mit $f$ \textbf{an} und \textbf{zeichnen} Sie die Tangente in das Diagramm \textbf{ein}.
        \textbf{Berechnen} Sie anschließend die Normale $n_f$ bei $x=1$ und \textbf{zeichnen} Sie diese ebenfalls in das Diagramm \textbf{ein}.
    \end{grayInfoBox}
    \begin{minipage}[t]{.5\textwidth}
        % \mbox{}\vspace{-2em}
        \mbox{}\\[1em]
        % $p$~(\IfSolutionT{\rlap{\color{red}1}}\qquad,\IfSolutionT{\rlap{\color{red}1.5}}\qquad)\\[2em]
        $p=\left(\tikz[baseline=-.1cm]{\node[inner sep=0pt, minimum width=1cm, minimum height=.5cm, draw]{\IfSolutionT{\color{red}1}};}, \tikz[baseline=-.1cm]{\node[inner sep=0pt, minimum width=1cm, minimum height=.5cm, draw]{\IfSolutionT{\color{red}1,5}};}\right)$\\[2em]
        $t_f=$~
        \IfSolutionT{\rlap{\raisebox{3pt}{\color{red}~$3x-1,5$}}}
        \dotfill\\[2em]
        $n_f=$~
        \IfSolutionT{\rlap{\raisebox{6pt}{\color{red}~$-\frac{1}{3}x+\frac{11}{6}$}}}
        \dotfill\\[1em]
        Platz für Rechnungen:
        \IfSolutionT{
            \par\color{red}
            \begin{align*}
                f'(x) & = 3x^2                                                                                                   \\
                f(1)  & = 3\cdot (1-1) + 1,5=1,5                                                                                 \\
                f'(1) & = 3\cdot 1^3=3                                                                                           \\
                t_f   & =f'(1)\cdot x + t=3\cdot x + (1,5-1\cdot 3)=\underline{\underline{3x-1.5}}                               \\
                n_f   & =-\frac{1}{f'(1)}\cdot x + t = -\frac{1}{3} \cdot x + \left(1,5-+1\cdot \left(-\frac{1}{3}\right)\right) \\
                      & =\underline{\underline{-\frac{1}{3}x+\frac{11}{6}}}
            \end{align*}
        }
    \end{minipage}
    \begin{minipage}[t]{.49\textwidth}
        \mbox{}\vspace*{-1em}\\
        \begin{tikzpicture}
            \centering
            \begin{axis}[
                xlabel=$x$,
                ylabel={$y$},
                axis lines=middle,
                axis line style={-{Triangle}{Bar}, thick},
                xmin=-\the\dimexpr\coordinatesystemradius pt + .5pt\relax,
                xmax=\the\dimexpr\coordinatesystemradius pt + .5pt\relax,
                ymin=-\the\dimexpr\coordinatesystemradius pt + .5pt\relax,
                ymax=\the\dimexpr\coordinatesystemradius pt + .5pt\relax,
                x=1cm,
                y=1cm,
                grid=both,
                minor tick num=1,
                xtick={-\coordinatesystemradius,...,\coordinatesystemradius},
                ytick={-\coordinatesystemradius,...,\coordinatesystemradius},
                every tick/.style={color=fgcolor},
                major grid style = {line width=.8pt},
                ]
                \addplot[smooth, samples=100, draw=accentcolor,very thick]{x^3+0.5};
                \IfSolutionT{
                \addplot[smooth, draw=red,very thick]{3 * (x-1)+1.5};
                \addplot[smooth, draw=green,very thick]{-(1/3) * (x-1)+1.5};
                \node [red] at (axis cs:.8,-.8){$t_f$};
                \node [green] at (axis cs:1.8,.7){$n_f$};
                \node [label={0:{(1,1.5)}},circle,fill,inner sep=2pt] at (axis cs:1,1.5){};
                }
            \end{axis}
        \end{tikzpicture}
        \captionof*{figure}{$f(x)=x^3+0,5$}
    \end{minipage}

    \vspace{2cm}
\end{task}
% Aufgabe 7
\begin{task}[points=2]{weitere Anwendungen der 1. Ableitung}
    \begin{grayInfoBox}
        Beantworten Sie die Folgenden Fragen in Ganzen Sätzen.
    \end{grayInfoBox}
    \begin{cpenumerate}[label=\alph*)]
        \item Was ist bei einem Vorzeichenwechsel der ersten Ableitung zu erwarten?
        \item Sei $f'(x)=2x$ ist $f(x)$ dann zwangsläufig positiv?
        % \item Was muss für die erste Ableitung eines Graphen gelten, wenn dieser punktsymmetrisch zum Ursprung ist?
    \end{cpenumerate}\mbox{}\\
    \mbox{}
    \IfSolutionT{\rlap{\raisebox{2pt}{\color{red}~a) Bei einem VZW der 1. Ableitung ist eine Extremstelle zu erwarten (Wendepunkt, Minimum oder Maximum)}}}
    \dotfill\\[2em]
    \mbox{}
    \IfSolutionT{\rlap{\raisebox{2pt}{\color{red}~b) Nein, denn beim Ableiten gehen Konstanten verloren, also gehen Verschiebungen in $y$ Richtung verloren.}}}
    \dotfill\\[2em]
    \mbox{}\dotfill\\[2em]
    \mbox{}\dotfill
\end{task}
\clearpage
% Aufgabe 8
\begin{task}[points=1]{Monotonieverhalten + rel. Extremwerte}
    \begin{subtask}[points=3]
        \begin{grayInfoBox}
            Beschreiben Sie das Monotonieverhalten der drei in den Graphen abgebildeten Funktionen.
        \end{grayInfoBox}
        \begin{figure}[H]
            \begin{subfigure}[t]{.33\textwidth}
                \centering
                \begin{tikzpicture}
                    \begin{axis}[
                        xlabel=$x$,
                        ylabel={$y$},
                        axis lines=middle,
                        axis line style={-{Triangle}{Bar}, thick},
                        xmin=-4.4,
                        xmax=4.4,
                        ymin=-4.4,
                        ymax=4.4,
                        x=0.6cm,
                        y=0.6cm,
                        grid=both,
                        minor tick num=1,
                        xtick={-4,...,4},
                        ytick={-4,...,4},
                        every tick/.style={color=fgcolor},
                        major grid style = {line width=.8pt},
                        ]
                        \addplot[thick,samples=100, smooth, blue]{0.7*x-2};%
                    \end{axis}
                \end{tikzpicture}
                \caption*{\\ \parbox{0.9\textwidth}{\dotfill}}
            \end{subfigure}
            \begin{subfigure}[t]{.33\textwidth}
                \centering
                \begin{tikzpicture}
                    \begin{axis}[
                        xlabel=$x$,
                        ylabel={$y$},
                        axis lines=middle,
                        axis line style={-{Triangle}{Bar}, thick},
                        xmin=-4.4,
                        xmax=4.4,
                        ymin=-4.4,
                        ymax=4.4,
                        x=0.6cm,
                        y=0.6cm,
                        grid=both,
                        minor tick num=1,
                        xtick={-4,...,4},
                        ytick={-4,...,4},
                        every tick/.style={color=fgcolor},
                        major grid style = {line width=.8pt},
                        ]
                        \addplot[thick,samples=100, smooth, orange]{-0.5*x+e^-x-2.2};%
                    \end{axis}
                \end{tikzpicture}
                \caption*{\\ \parbox{0.9\textwidth}{\dotfill}}
            \end{subfigure}
            \begin{subfigure}[t]{.33\textwidth}
                \centering
                \begin{tikzpicture}
                    \begin{axis}[
                        xlabel=$x$,
                        ylabel={$y$},
                        axis lines=middle,
                        axis line style={-{Triangle}{Bar}, thick},
                        xmin=-4.4,
                        xmax=4.4,
                        ymin=-4.4,
                        ymax=4.4,
                        x=0.6cm,
                        y=0.6cm,
                        grid=both,
                        minor tick num=1,
                        xtick={-4,...,4},
                        ytick={-4,...,4},
                        every tick/.style={color=fgcolor},
                        major grid style = {line width=.8pt},
                        ]
                        \addplot[thick,samples=100, smooth, green]{0.1*x^3};%
                    \end{axis}
                \end{tikzpicture}
                \caption*{\\ \parbox{0.9\textwidth}{\dotfill}}
            \end{subfigure}
        \end{figure}
    \end{subtask}
    \begin{subtask}[points=8]
        \begin{grayInfoBox}
            Ermitteln Sie für die Extremwerte der folgenden Funktionen. Legen Sie Ihren Rechenweg dabei \textbf{nachvollziehbar} dar.
        \end{grayInfoBox}
        \begin{cpenumerate}
            \item $f(x)=8x^2-25$
            \item $g(x)=x^4-2x^2+2$
            \item $h(x)=x^3+3x^2-a$
            \item $i(t)=a^5+b^4-t^2+6t-8$
        \end{cpenumerate}
    \end{subtask}
\end{task}
\clearpage
% Aufgabe 9
\begin{task}[points=1]{Evtl. Krümmungsverhalten und Wendepunkte}
    \begin{grayInfoBox}
        Gegeben Sei die Funktion $f(x):=\frac{1}{3}x^3-2x^2+3x+1$. \textbf{Berechnen} Sie alle Wendepunkte von $G_f$.
        Geben Sie den \textbf{vollständigen} Rechenweg an. Sie können zur Hilfe $G_f$ in das Koordinatensystem einzeichnen.
        Geben Sie anschließend das Krümmungsverhalten im Bereich $[-1;4]$ an
    \end{grayInfoBox}
    \def\coordinatesystemradius{4}
    \begin{minipage}[t]{.49\textwidth}
        Platz für die Rechnung:
        \IfSolutionT{
            \textcolor{red}{
                \begin{align*}
                    f'(x)   & =x^2-4x+3                                                     \\
                    f''(x)  & =2x-4                                                         \\
                    f'''(x) & =2                                                            \\
                    f''(x)  & =0=2x-4\Rightarrow x=2                                        \\
                    f'''(2) & =2\neq 0\Rightarrow\text{Wendepunkt}                          \\
                    f(2)    & =\frac{8}{3} \Rightarrow\text{Wendepunkt bei }(2,\frac{8}{3}) \\
                            & \text{Rechtsgekrümmt für } x\in [-1;2[,                       \\
                            & \text{Linksfekrümmt für } x\in ]2;4]
                \end{align*}}
        }
    \end{minipage}
    \begin{minipage}[t]{.5\textwidth}
        \centering\mbox{}\\
        \begin{tikzpicture}
            \centering
            \begin{axis}[
                xlabel=$x$,
                ylabel={$y$},
                axis lines=middle,
                axis line style={-{Triangle}{Bar}, thick},
                xmin=-\the\dimexpr\coordinatesystemradius pt + .5pt\relax,
                xmax=\the\dimexpr\coordinatesystemradius pt + .5pt\relax,
                ymin=-\the\dimexpr\coordinatesystemradius pt + .5pt\relax,
                ymax=\the\dimexpr\coordinatesystemradius pt + .5pt\relax,
                x=1cm,
                y=1cm,
                grid=both,
                minor tick num=1,
                xtick={-\coordinatesystemradius,...,\coordinatesystemradius},
                ytick={-\coordinatesystemradius,...,\coordinatesystemradius},
                every tick/.style={color=fgcolor},
                major grid style = {line width=.8pt},
                ]
                \IfSolutionT{
                \addplot[smooth, samples=100, draw=red,very thick]{(1/3)*x^3-2*x^2+ 3*x + 1};
                \node [red] at (axis cs:-1,-.8){$G_f$};
                \node [label={0:{$\left(2,\frac{8}{3}\right)$}},circle,fill,inner sep=2pt] at (axis cs:2,1.666){};
                }
            \end{axis}
        \end{tikzpicture}
        \captionof*{figure}{$G_f$}
    \end{minipage}
\end{task}
% \section*{Stochastik}
% % Aufgabe 10
% \begin{task}[points=1]{relative und absolute Häufigkeiten berechnen}

% \end{task}
% % Aufgabe 11
% \begin{task}[points=1]{Häufigkeiten in eine Vierfeldertafel eintragen}

% \end{task}
% % Aufgabe 12
% \begin{task}[points=1]{Rechnen mit Wahrscheinlichkeiten}

% \end{task}
% % Aufgabe 13
% \begin{task}[points=1]{Baumdiagramm aufstellen}

% \end{task}
% % Aufgabe 14
% \begin{task}[points=1]{Pfadregeln anwenden}

% \end{task}
% % Aufgabe 15
% \begin{task}[points=1]{Auf stochastische Unabhängigkeit prüfen}

% \end{task}
% % Aufgabe 16
% \begin{task}[points=1]{bedingte Wahrscheinlichkeiten berechnen}

% \end{task}
% % Aufgabe 17
% \begin{task}[points=1]{Ergebnisse im Sachzusammenhang interpretieren}

% \end{task}
\end{document}
